\documentclass[11pt]{article}
\usepackage{amsfonts,amssymb,amsthm,eucal,amsmath}
\usepackage{graphicx}
\usepackage[T1]{fontenc}
\usepackage{latexsym,url}
\usepackage{array}
\usepackage{subfig}
\usepackage{comment}
\usepackage{color}
\usepackage{hyperref}

\newcommand{\myspace}{\vspace{.1in}\noindent}
\newcommand{\mymyspace}{\vspace{.1in}}
\usepackage[inner=30mm, outer=30mm, textheight=225mm]{geometry}

\newtheorem{theorem}{Theorem}[section]
\newtheorem{prop}[theorem]{Proposition}
\newtheorem{corollary}[theorem]{Corollary}
\newtheorem{defn}[theorem]{Definition}
\newtheorem{notn}[theorem]{Notation}
\newtheorem{cond}[theorem]{Condition}
\newtheorem{ex}[theorem]{Example}
\newtheorem{rmk}[theorem]{Remark}
\newcommand{\co}{\negthinspace :}
\newcommand{\N}{\mathbb{N}}
\newcommand{\Z}{\mathbb{Z}}
\newcommand{\R}{\mathbb{R}}
\newcommand{\C}{\mathbb{C}}
\newcommand{\CP}{\mathbb{CP}}
\newcommand{\PSL}{\mathrm{PSL}_2(\mathbb{C})}
\newcommand{\area}{\operatorname{area}}
\newcommand{\diag}{\operatorname{diag}}
\newcommand{\nt}{\negthinspace}
\newcommand{\TODO}{{\color{red} TODO}}

\title{A linear time direct solver for convex tridiagonal quadratic programs with bound constraints}
\author{Geoffrey Irving\thanks{Email: irving@naml.us, Otherlab, San Francisco, CA, United States}}
\date{Version 1, \today}

\begin{document}
\maketitle

\begin{abstract}
We present a linear time direct solver for symmetric positive definite tridiagonal quadratic programming with bound constraints.  Our algorithm is similar in structure to the
algorithm for linear time shortest paths in triangulated simple polygons by Lee and Preparata.  The algorithm is generalizable to other nonlinear convex programs of similar
structure, though we do not know of any such examples.
\end{abstract}

\section{Background}

Consider a symmetric positive definite tridiagonal quadratic program with bound constraints of the form
\begin{align} \label{qp}
\begin{array}{cc}
\min          \qquad& \frac{1}{2} x^T A x + b^T x \\
\textrm{s.t.} & l \le x \le u
\end{array}
\end{align}
where $x,l,u \in \R^n$ and $A$ is symmetric positive definite tridiagonal.  The problem is feasible iff $l \le u$, and can be solved efficiently using an interior point method (see \cite{gondzio2012interior}
for a survey).  Each Newton iteration requires only an $O(n)$ tridiagonal direct solve; since interior point methods typically take between $O(1)$ and $O(\log n)$ iterations to converge \cite{colombo2008further},
this results in a quite fast method overall.  However, this peformance is not guaranteed, and any iterative method will require more iterations if greater accuracy is required.

In this paper, we present a linear time direct solver specialized for problems of the form (\autoref{qp}).  We discovered this algorithm by accident: we intended to derive an linear time algorithm
for shortest paths (geodesics) in triangle strips immersed in three dimensions, but mistakenly wrote down a sum of squared lengths instead of the correct sum of unsquared lengths.  We then arrived at a linear time algorithm
using geometric intuition before realizing that our motivating problem was not a quadratic program after all.  In fact, essentially the same algorithm works in both cases, although an extra data structure is required
for quadratic programming as discussed below.  The shortest path case is due to Lee and Preparata \cite{lee1984euclidean}, but to our knowledge its application to quadratic programming is unpublished.

We review the geometric case first.  Consider a sequence of segments $p_i q_i$ with $p_i, q_i \in \R^2$ for $i = 0, \ldots, n$, forming a quadrilateral strip.  Place an interpolated point
$r_i = (1-t_i) p_i + t_i q_i$ on each segment, where $0 \le t_i \le 1$.  Assume that the first and last points are fixed, so that $p_0 = q_0$ and $p_n = q_n$, and that no two segments intersect.
We seek to minimize the total length
\begin{align*}
L &= \sum_{i=0}^{n-1} \left\| r_i - r_{i+1} \right\|.
\end{align*}
The algorithm proceeds from $k = 0$ to $k = n-1$, where stage $k$ minimizes the length up to $r_k$ for the two extreme values $r_k = p_k, q_k$.  Since segments do not intersect, the optimal value
of $t_i$ for $i < k$ is a monotonic function of $x_k$, so the optimal path for any intermediate value $0 < t_k < 1$ lies between the optimal paths for $t_k = 0,1$.  Thus, our intermediate state has the form shown in
\autoref{kite}: the two paths stay together for a certain distance, then curve apart in opposite directions.  Moving from $k$ to $k+1$ consists of adding new edges to each path, then eroding until
we restore convexity.  To determine whether to remove a vertex at index $j$ between $i$ and $k$, we compute a shortest path between $r_i$ and $r_k$ and compare it to the bounds on $r_j$, which takes $O(1)$ time.
For full details, see \cite{lee1984euclidean}.

\section{Tridiagonal quadratic programs}

On to the quadratic program (\autoref{qp}).  We first use the Cholesky decomposition $A = L L^T$ to convert the quadratic energy into a sum of energies of adjacent variables:
\begin{align*}
E &= \frac{1}{2} x^T A x + b^T x  \\
  &= \frac{1}{2} x^T L L^T x + b^T x \\
  &= \frac{1}{2} \left\|L^T x + L^{-1}b \right\|^2 - \frac{1}{2} b^T L^{-T} L^{-1} b \\
  &= \sum_{i=0}^{n-1} E_{i,i+1}(x_i,x_{i+1}) + \textrm{const}
\end{align*}
using
\begin{align} \label{quadratic}
E_{i,i+1}(x_i,x_{i+1}) = \frac{1}{2} (L_{i,i} x_i + L_{i+1,i} x_{i+1} + c_i)^2
\end{align}
where we have introduced sentinels $x_{n+1} = L_{n+1,n} = 0$ for notational simplicity.  We then sweep through adjusting the signs of each $x_i$ so that $L_{i+1,i} \le 0$; since $L_{i,i} > 0$, this
ensures that the local optimal value of $x_i$ is a weakly increasing function of $x_{i-1}$ and $x_{i+1}$.

For any $0 \le i < k \le n$, we define a function $E_{i,k}$ by partially minimizing over all variables between $x_i$ and $x_k$:
$$E_{i,k}(x_i,x_k) = \min_{x_{i+1}, \ldots, x_{k-1}} \sum_{i \le j < k} E_{j,j+1}(x_j,x_{j+1}).$$
This minimization produces a linear system for the intermediate $x_j$, and substituting a linear function into a quadratic gives a quadratic.  Thus each $E_{i,k}$ is of the form (\autoref{quadratic}) up to a constant,
and we can combine $E_{i,j}$ and $E_{j,k}$ into $E_{i,k}$ in $O(1)$ time.  $E_{i,k}$ encodes the set of optimal paths from $x_i$ to $x_k$ when bound constraints are ignored.  Note that $E_{0,n} = E$
is the entire energy function up to a constant.

As in the geometric case, our algorithm proceeds from $k = 0$ to $k = n-1$, maintaining two minimizers of the partial energy up to $x_k = l_k$ and $x_k = u_k$.  We represent the two partial minimizers in terms of
their active constraints.  The $x_k = l_k$ and $x_k = u_k$ paths will agree up to a certain $x_j$, then deviate
in opposite directions with the $x_k = l_k$ path touching only $l_i$ constraints and the $x_k = u_k$ path touching only $u_i$ constraints.  To advance from $k$ to $k+1$, we extend the two partial solutions by adding
$l_{k+1}$ and $u_{k+1}$ constraints, respectively, then remove constraints that violate local optimality starting from $l_k$ and $u_k$ until optimality is restored (\autoref{erode}a).  In the process we may erode the entire separate
portion of one of our two paths, in which case we continue eroding the beginning of the other separate path, extending the common section of the paths in the process (\autoref{erode}b).
A constraint is considered for removal only when a neighbor is added or removed, and a constraint is added (resp\. removed) at most once.  Thus, as long as we know the appropriate $E_{ij}$ when we need them, each step
is amortized constant time, and the overall complexity is $O(n)$.

Which $E_{ij}$ do we need?  Let the lower and upper paths have indices $(i_0, i_1, \ldots, i_a)$ and $(j_0, j_1, \ldots, j_b)$, respectively, where $i_0 = j_0$ is the last shared vertex of the paths
and $i_a = j_b = k$ is the index just added.  When eroding the lower path from the left, we need energy terms $E_{i_{a-2}i_{a-1}}$ and $E_{i_{a-1}i_{a}}$, which are available as long as we maintain
energy terms between all adjacent indices $E_{i_0i_1}, E_{i_1i_2}, \ldots$ along the path.  When a new index is appended on the left we add a new energy term along with it, and if index $i_{a-1}$ is
eliminated we reduce $E_{i_{a-2}i_{a-1}}$ and $E_{i_{a-1}i_{a}}$ into $E_{i_{a-2},i_a}$ in $O(1)$ time.

If we erode the entire lower path down to $(i_0,i_a)$ and start eroding the upper path $(j_0, j_1, \ldots, j_b)$, our decision depends on the energy terms $E_{j_0j_1}$ and $E_{j_1i_a} = E_{j_1j_b}$.
Thus, in addition to consecutive energy terms we need the term $E_{j_1j_b}$ crossing the entire disjoint portion of the path.  This amounts to maintaining the product of a queue of semigroup elements,
and we use a modified version of the ``two stack'' solution of sevenkplus\footnote{CS Theory Stack Exchange, \url{http://cstheory.stackexchange.com/questions/18655}}.  Specifically, we pick a pivot
index $j_1 \le p \le j_b$, store $E_{j p}$ for $j < p$, and store $E_{p j_b}$.  $E_{j_1 j_b}$ is then available by combining $E_{j_1 p}$ with $E_{p j_b}$.  When a new element is appended to the left
we update $E_{p j_b}$.  If index $j_1$ is eliminated nothing need be done unless $p < j_2$, in which case we reset the pivot to $p = j_b$ and recompute all $E_{j p}$ for $j < p$ in reverse order.
This cost is amortized $O(1)$ time by charging the computation of $E_{j p}$ to the append operation which added $j$.  Note that the pivot index is not necessarily part of the path.

\section{Other nonlinear programs}

Expressed in terms of energy functions, our algorithm solves the following optimization problem:
\begin{align*}
\begin{array}{cc}
\min          \qquad& \sum E_{i,i+1}(x_i,x_{i+1}) \\
\textrm{s.t.} & l \le x \le u
\end{array}
\end{align*}
Define $E_{ij}$ as above.  The algorithm works and runs in $O(n)$ time if the following properties hold:
\begin{enumerate}
\item For any $i < j < k$, we can combine $E_{ij}$ and $E_{jk}$ into $E_{ik}$ in $O(1)$ time.
\item Given $E_{ij}$, $E_{jk}$, $x_i$, and $x_k$, we can compute the optimum value of $x_j$ in $O(1)$ time.
\item For $i < j < k$, $x_j$ is a monotonic function of $x_i$ and $x_k$.
\end{enumerate}
Unfortunately, these properties are quite fragile, and we do not know of any useful examples beyond shortest paths in polygons and tridiagonal quadratic programming.  For example, if the index of
refraction is allowed to vary across quadrilaterals in a quad strip, we can no longer combine $E_{ij}$ with $E_{jk}$ in $O(1)$ time, and the algorithm fails.

\bibliography{references}
\bibliographystyle{acm}

\end{document}
